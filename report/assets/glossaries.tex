\newglossaryentry{VoIP}
{
    name=VoIP,
    description={VoIP staat voor ``Voice over Internet Protocol'' en is een technologie waarmee spraak- en multimedia-communicatie via internet mogelijk is. Het maakt gebruik van digitale signalen om spraak en andere gegevens te verzenden in plaats van traditionele telefoonlijnen}
}

\newglossaryentry{SSOT}
{
    name=SSOT,
    description={SSOT staat voor ``Single Source of Truth'' en verwijst naar het concept waarbij er één enkele, centrale bron van waarheid is voor gegevens in een organisatie. Dit zorgt voor consistentie en nauwkeurigheid van gegevens in verschillende systemen en processen}
}

\newglossaryentry{holacratie}
{
    name=holacratie,
    description={Holacratie is een organisatiestructuur waarbij de nadruk ligt op gedecentraliseerde besluitvorming en autonomie van individuele cirkels en rollen. Het streeft naar flexibiliteit en snelle aanpassing in een organisatie door traditionele hiërarchieën te vermijden}
}

\newglossaryentry{constitutie}
{
    name=constitutie,
    description={De constitutie van holacratie is een soort besturingssysteem voor een holacratisch bedrijf. Het bevat de regels en processen die worden gebruikt om de organisatie te besturen en te beheren}
}

\newglossaryentry{Scrum}
{
    name=Scrum,
    description={Scrum is een agile projectmanagementmethode die wordt gebruikt voor softwareontwikkeling en andere projecten. Het is gebaseerd op iteratieve en incrementele processen, waarbij teams regelmatig bijeenkomen om taken te bespreken en prioriteiten te stellen}
}

\newglossaryentry{Agile}
{
    name=Agile,
    description={Agile is een manier van iteratief en incrementeel ontwikkelen, waarbij projecten worden opgedeeld in kleine stukjes om snel aan veranderende eisen te kunnen voldoen}
}

\newglossaryentry{sprint}
{
    name=sprint,
    description={Een sprint is een korte, afgebakende periode binnen een agile ontwikkelingsproces waarin een team zich richt op het voltooien van een specifieke reeks taken of functies. Sprints duren meestal twee tot vier weken en zijn bedoeld om meetbare vooruitgang te boeken}
}

\newglossaryentry{product owner}
{
    name=product owner,
    description={De product owner is een rol binnen een agile ontwikkelingsteam die verantwoordelijk is voor het beheren van de product backlog, het definiëren van functies en prioriteiten, en het vertegenwoordigen van de behoeften van de klant}
}

\newglossaryentry{CI/CD}
{
    name=CI/CD,
    description={CI/CD staat voor ``Continuous Integration'' en ``Continuous Delivery'' (of ``Continuous Deployment'') en verwijst naar een reeks praktijken en hulpmiddelen die worden gebruikt in softwareontwikkeling om automatische tests en continue levering van software te bevorderen}
}

\newglossaryentry{OpenAPI}
{
    name=OpenAPI,
    description={OpenAPI is een specificatie die wordt gebruikt om RESTful API's te beschrijven en documenteren. Het biedt een gestandaardiseerde manier om de functionaliteit en endpoints van een API te definiëren en te delen}
}

\newglossaryentry{OpenAPI-specificatie}
{
    name=OpenAPI-specificatie,
    description={De OpenAPI-specificatie is een gestandaardiseerde manier om de functionaliteit, endpoints en vereisten van een RESTful API te beschrijven en te documenteren. Het maakt het gemakkelijk om API-documentatie te genereren en API's te integreren}
}

\newglossaryentry{ORM}
{
    name=ORM,
    description={ORM staat voor ``Object-Relational Mapping'' en is een techniek die wordt gebruikt in softwareontwikkeling om code te koppelen aan relationele databases. Het maakt het gemakkelijker om gegevens te een database te manipuleren en te beheren, en vereist geen handmatig geschreven SQL-query's}
}

\newglossaryentry{Preact}
{
    name=Preact,
    description={Preact is een JavaScript-bibliotheek voor het bouwen van gebruikersinterfaces en webtoepassingen. Het is een lichtgewicht alternatief voor React en biedt vergelijkbare functionaliteit voor het ontwikkelen van interactieve webpagina's}
}

\newglossaryentry{Tailwinds}
{
    name=Tailwinds,
    description={Tailwinds is een open-source CSS-framework dat wordt gebruikt voor het bouwen van webtoepassingen. Het is een alternatief voor handmatig CSS regels te schrijven. Het biedt een set van voorgedefinieerde klassen en hulpmiddelen om de ontwikkeling van webpagina's te versnellen en aan te passen.}
}

% Niet gebruikt
\newglossaryentry{OAuth}
{
    name=OAuth,
    description={OAuth is een open standaard voor autorisatie die wordt gebruikt voor het veilig delen van gegevens tussen applicaties, zonder dat gebruikers hun wachtwoord hoeven te delen. Het wordt vaak gebruikt voor API-authenticatie en toegangsbeheer}
}

% Niet gebruikt
\newglossaryentry{OAuth-token}
{
    name=OAuth token,
    description={Een OAuth-token is een beveiligingsreferentie die wordt gebruikt om de identiteit en autorisatie van een gebruiker of applicatie te verifiëren bij het communiceren met een beveiligde API of service. Het wordt gegenereerd na een succesvolle OAuth-authenticatie}
}

\newglossaryentry{Go}
{
    name=Go,
    description={Go, ook wel bekend als Golang, is een programmeertaal ontwikkeld door Google. Het staat bekend om zijn eenvoud en efficiëntie en wordt vaak gebruikt voor bijvoorbeeld API's, web servers en cloudservices}
}

\newglossaryentry{API}
{
    name=API,
    description={API staat voor ``Application Programming Interface'' en verwijst naar een set regels en protocollen die wordt gebruikt om softwaretoepassingen met elkaar te laten communiceren. Het definieert hoe verschillende softwarecomponenten met elkaar kunnen interageren}
}

\newglossaryentry{API-sleutel}
{
    name=API-sleutel,
    description={Een API-sleutel is een unieke code of sleutel die wordt verstrekt aan ontwikkelaars om toegang te krijgen tot een API of service. Het wordt gebruikt voor authenticatie en autorisatie en identificeert de bron van het verzoek}
}

\newglossaryentry{frontend}
{
    name=frontend,
    description={Een frontend verwijst naar het deel van een softwaretoepassing dat zichtbaar is voor de gebruiker. Dit omvat meestal de gebruikersinterface van een applicatie}
}

\newglossaryentry{server}
{
    name=server,
    description={Een server is een programma of apparaat dat verzoeken verwerkt en diensten levert aan andere programma's of apparaten op een netwerk. Het kan verschillende soorten diensten aanbieden, waaronder het hosten van websites, API's, het verwerken van e-mails en meer}
}

\newglossaryentry{ERD}
{
    name=ERD,
    description={ERD staat voor ``Entity-Relationship Diagram'' en is een visuele representatie van de entiteiten, relaties en attributen in een database}
}

% Niet gebruikt
\newglossaryentry{YAML}
{
    name=YAML,
    description={YAML staat voor ``YAML Ain't Markup Language'' (of ``YAML is geen opmaaktaal'') en is een opmaaktaal die wordt gebruikt voor het configureren van gegevens in een menselijk leesbare en schrijfbare indeling. Het wordt vaak gebruikt voor configuratiebestanden en gegevensuitwisseling tussen programma's}
}

\newglossaryentry{REST}
{
    name=REST,
    description={REST staat voor ``Representational State Transfer'' en is een architectuurstijl voor het ontwerpen van netwerksystemen. Het gebruik van standaard HTTP-methoden voor gegevensmanipulatie, zoals GET, POST, PUT en DELETE}
}

% Niet gebruikt
\newglossaryentry{CRUD}
{
    name=CRUD,
    description={CRUD staat voor ``Create, Read, Update, Delete'' en verwijst naar de vier basiselementaire bewerkingen voor gegevensmanipulatie in een database of systeem. Het omvat het maken, lezen, bijwerken en verwijderen van gegevensrecords}
}

\newglossaryentry{scope creep}
{
    name=scope creep,
    description={Scope creep verwijst naar het fenomeen waarbij de omvang van een project ongecontroleerd toeneemt tijdens de ontwikkeling. Dit kan leiden tot vertragingen, budgetoverschrijdingen en andere problemen}
}

\newglossaryentry{Salt}
{
    name=Salt,
    description={Salt is een automatiseringstool en framework voor het maken en bijwerken van systemen. Het wordt vaak gebruikt voor het beheren van configuraties en het anderzijds instellen van software op meerdere servers}
}

\newglossaryentry{systemd-journal}
{
    name=systemd-journal,
    description={systemd-journal is een systeemlogboekservice die wordt gebruikt voor het verzamelen en opslaan van logboeken van verschillende processen en services. Het wordt vaak gebruikt in combinatie met systemd, een systeem- en servicebeheerder voor Linux}
}

\newglossaryentry{Promtail}
{
    name=Promtail,
    description={Promtail is een logboekverzamelaar die wordt gebruikt voor het verzamelen, filteren en verzenden van logboeken naar een centrale server}
}

\newglossaryentry{Loki}
{
    name=Loki,
    description={Loki is een logberichtaggregator die wordt gebruikt voor het verzamelen en opslaan van logberichten}
}

\newglossaryentry{Grafana}
{
    name=Grafana,
    description={Grafana is een open-source analyse- en monitoringplatform dat wordt gebruikt voor het visualiseren van gegevens uit verschillende bronnen}
}

\newglossaryentry{Sentry}
{
    name=Sentry,
    description={Sentry is een open-source foutenbewaking en rapportageplatform dat wordt gebruikt voor het verzamelen en analyseren van fouten en foutmeldingen}
}

\newglossaryentry{OIDC}
{
    name=OIDC,
    description={OIDC staat voor ``OpenID Connect'' en is een open standaard voor authenticatie die wordt gebruikt voor het verifiëren van de identiteit van een gebruiker. Het is gebaseerd op het OAuth 2.0-protocol. Met OIDC kan bijvoorbeeld ingelogd worden met Google of Facebook op een website, in plaats van zelf een accout te moeten registreren}
}
