\documentclass[../report.tex]{subfiles}

\begin{document}

% Introductie onderwerp, aanleiding en organisatie
In de moderne wereld van bedrijfsvoering is de beschikbaarheid van gegevens van cruciaal belang voor het nemen van weloverwogen beslissingen. ICT-bedrijven werken vaak met vele \glspl*{server} en nog meer virtuele machines. Zonder een overzicht van deze systemen en hun onderdelen kunnen beslissingen niet grondig beargumenteerd worden. Misverstanden kunnen ontstaan, onduidelijkheid volgt en onnodige kosten kunnen worden gemaakt.

Deze hierboven benoemde situatie is geen onbekend probleem bij Voys. Voys is een telecommunicatiebedrijf opgericht in Groningen. Het zorgt dat bedrijven kunnen bellen en bereikbaar zijn voor hun klanten. Dit bereikt het bedrijf met behulp van \gls*{VoIP} in de cloud. Het is een holacratisch bedrijf, wat betekent dat het een vorm van zelfsturing heeft. Medewerkers moeten rapporteren aan, noch luisteren naar een baas (want die is er niet), maar kunnen vrij werken binnen de spelregels van de \gls*{constitutie} van \gls*{holacratie} \parencite{holacracy}.

Om de eerder genoemde problemen te kunnen voorkomen is een zogeheten ``single source of truth'' (afgekort \gls*{SSOT}) behulpzaam, waarin relevante gegevens van een bedrijf zijn weer te geven. Het is niet vaak nodig, maar voor de keren dat dit wel het geval is, is het van groot belang.

% Opdracht en doelstelling
Daarom is er een opdracht opgesteld om een dergelijk programma te schrijven. Het doel van het programma is om een overzicht te geven van hardware en netwerkinstellingen van de \glspl*{server} van Voys.

% Onderzoek
Bij het uitvoeren van de afstudeeropdracht is er ondersteunend onderzoek uitgevoerd. Het doel van dit onderzoek is om te ontdekken of gebruikers van de applicatie veelvoorkomende doeleinden kunnen uitvoeren. Met de resultaten kunnen vervolgens verbeteringen worden doorgevoerd waar dat nodig is.

% Vooruitblik
Dit verantwoordingsverslag beschrijft de aanpak en resultaten van het afstudeerproject. In dit verslag komen de volgende hoofdstukken voor: projectaanpak en risico's, analyse van de requirements, ontwerp, realisatie, deployment, validatie en beheer, aantoning van HBO-I competenties, de conclusie en lessons learned. Het doel van dit verslag is om een duidelijk beeld te geven van het project en de stappen die zijn gezet om het tot een succesvol einde te brengen.

\end{document}
