\documentclass[../report.tex]{subfiles}

\begin{document}

Bij het projectplan voor het project is een lijst van competenties opgesteld die bij het afstuderen aangetoond worden \parencite{projectplan}. Deze vaardigheden komen gedeeltelijk uit de HBO-i domeinomschrijving die opgesteld is door de HBO-i stichting. Andere competenties komen uit de handleiding voor het afstudeerproject van de \textcite{handleiding_afstudeerproject}. In dit hoofdstuk wordt per vaardigheid beschreven hoe deze aangetoond is.

\section{De competenties}

Met de analyseren-competentie wordt aangetoond dat ``[\dots] het analyseren van processen, producten en informatiestromen in hun onderlinge samenhang en context [behelst wordt].'' \parencite{hbo_i}. Deze competentie is aangetoond door reeds bestaande software zoals NetBox en GoCollect to bestuderen en te analyseren in \autoref{sec:alternatieven} en \autoref{ch:realization}. Deze competentie is verder aangetoond door risico's te analyseren in \autoref{ch:approach_and_risks}, en door een analyse te maken van de user stories in \autoref{ch:requirements}.

Bij de ontwerpcompetentie wordt aangetoond dat ``[\dots] het ontwerpen van een (deel van een) ICT-systeem [behelst wordt] op basis van specificaties.'' \parencite{hbo_i} De ontwerpcompetentie is aangetoond door in \autoref{ch:design} een ontwerp te maken voor de softwarearchitectuur. Er waren geen randvoorwaarden opgesteld, maar er is wel rekening gehouden met de opgestelde requirements en user stories. Alternatieve oplosrichtingen voor de afstudeeropdracht waren het ontwikkelen van een applicatie die tussen GoCollect en NetBox zou kunnen werken als middleware, of het forken van GoCollect zodat het direct met NetBox zou kunnen werken. Er is gekozen voor de huidige oplossingsrichting omdat het de meeste vrijheid bood om aan de user stories en requirements te voldoen. Er is ook een overweging gemaakt voor het kiezen van \gls*{OIDC} in \autoref{ch:realization}.

De realisatiecompetentie omvat volgens de \textcite{hbo_i} ``[\dots] het realiseren van een (deel van een) ICT-systeem op basis van een ontwerp.'' Dit is aangetoond door de beroepsproducten te ontwikkelen zoals omschreven in \autoref{ch:realization} op basis van het ontwerp omschreven in \autoref{ch:design}. De keuze van de taal wordt ook besproken in \autoref{ch:realization}. De applicatie voldoet niet aan alle user stories en requirements, maar uit het onderzoek bleek dat de user stories die wel geïmplementeerd zijn met succes uitgevoerd kunnen worden \parencite{research_report}.

Projectmatig werken betekent dat er ``een plan ten grondslag aan de uitvoer van het afstudeerproject moet liggen'' \parencite{handleiding_afstudeerproject}. De competentie is aangetoond door het plan dat is opgesteld in \autoref{ch:approach_and_risks} toe te passen tijdens de ontwikkeling van de beroepsproducten.

De competentie ``onderzoekend vermogen'' is aangetoond door een praktijkonderzoek uit te voeren naar in welke mate de SSOT-applicatie voldoet aan de opgestelde user stories \parencite{research_report}.

Met de schriftelijke communicatie-competentie wordt aangetoond dat er professioneel gecommuniceerd kan worden met collega's en begeleiders. Schriftelijke communicatie vond plaats op de volgende manieren:

\begin{itemize}
    \item Met het communicatiemiddel \textit{Slack}, met collega's en begeleiders.
    \item Met e-mail, met collega's en begeleiders.
    \item Met \textit{Microsoft Teams}, met de toenmalige docentbegeleider.
\end{itemize}

Bewijs voor aantoning van deze competentie kan gevonden worden in het feedbackformulier professioneel handelen \parencite{feedbackformulier_professioneel_handelen}.

\end{document}
