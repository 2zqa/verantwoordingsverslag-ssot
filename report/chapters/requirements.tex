\documentclass[../report.tex]{subfiles}

\begin{document}

Een product kan niet gemaakt worden zonder requirements. Requirements zijn de specificaties van een product. In dit hoofdstuk wordt besproken hoe de requirements zijn verkregen en verwerkt.


\section{Stakeholders}

Bij het project zijn drie stakeholders geïdentificeerd. Alle drie de stakeholders hebben holacratische rollen binnen de infrastructuurcirkel. Twee van de drie stakeholders zijn \gls*{product owner} en bedrijfsbegeleider. De derde stakeholder had extra wensen voor hun gebruik van de SSOT.

\section{Verzamelen}

Alle stakeholders zijn geïnterviewd over hoe ze verwachten de app te willen gebruiken. Hun werd eerst uitgelegd wat user stories zijn, en ze zijn vervolgens gevraagd om hun wensen te formuleren in dit formaat. Deze zijn genoteerd in een apart document \parencite{user_stories}. Verdere requirements zijn verkregen door wekelijkse gesprekken met de \gls*{product owner}, en een paar requirements zijn verkregen door informele, ongestructureerde gesprekken met stakeholders. Tenslotte zijn de user stories omgezet tot requirements in de vorm van incrementen in de backlog \parencite{backlog}.

Behalve user stories zijn er ook requirements omtrent de te verzamelen specificaties verkregen. Deze zijn verzameld door interviews en gedeeltelijk door informele, ongestructureerde gesprekken met stakeholders. Deze requirements zijn ook omgezet tot incrementen en in de backlog gezet.

\section{Analyseren}

De user stories die uit de interviews zijn gekomen zijn gelezen en bleken niet tegenstrijdig te zijn met elkaar. Requirements over de specificaties zijn met een MoSCoW-analyse geprioriteerd door de \gls*{product owner}. Verdere prioritering van incrementen zijn wekelijks bijgesteld bij meetings met de \gls*{product owner}.

\end{document}
